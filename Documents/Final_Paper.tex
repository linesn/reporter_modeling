
\documentclass{amsart}
%%%%%%%%%%%%%%%%%%%%%%%%%%%%%%%%%%%%%%%%%%%%%%%%%%%%%%%%%%%%%%%%%%%%%%%%%%%%%%%%%%%%%%%%%%%%%%%%%%%%%%%%%%%%%%%%%%%%%%%%%%%%%%%%%%%%%%%%%%%%%%%%%%%%%%%%%%%%%%%%%%%%%%%%%%%%%%%%%%%%%%%%%%%%%%%%%%%%%%%%%%%%%%%%%%%%%%%%%%%%%%%%%%%%%%%%%%%%%%%%%%%%%%%%%%%%
\usepackage{amsfonts}

\setcounter{MaxMatrixCols}{10}
%TCIDATA{OutputFilter=LATEX.DLL}
%TCIDATA{Version=5.50.0.2960}
%TCIDATA{<META NAME="SaveForMode" CONTENT="1">}
%TCIDATA{BibliographyScheme=Manual}
%TCIDATA{Created=Thursday, May 12, 2022 21:29:02}
%TCIDATA{LastRevised=Thursday, May 12, 2022 21:51:57}
%TCIDATA{<META NAME="GraphicsSave" CONTENT="32">}
%TCIDATA{<META NAME="DocumentShell" CONTENT="Articles\SW\AMS Journal Article">}
%TCIDATA{CSTFile=amsartci.cst}

\newtheorem{theorem}{Theorem}
\theoremstyle{plain}
\newtheorem{acknowledgement}{Acknowledgement}
\newtheorem{algorithm}{Algorithm}
\newtheorem{axiom}{Axiom}
\newtheorem{case}{Case}
\newtheorem{claim}{Claim}
\newtheorem{conclusion}{Conclusion}
\newtheorem{condition}{Condition}
\newtheorem{conjecture}{Conjecture}
\newtheorem{corollary}{Corollary}
\newtheorem{criterion}{Criterion}
\newtheorem{definition}{Definition}
\newtheorem{example}{Example}
\newtheorem{exercise}{Exercise}
\newtheorem{lemma}{Lemma}
\newtheorem{notation}{Notation}
\newtheorem{problem}{Problem}
\newtheorem{proposition}{Proposition}
\newtheorem{remark}{Remark}
\newtheorem{solution}{Solution}
\newtheorem{summary}{Summary}
\numberwithin{equation}{section}

\input{tcilatex}

\begin{document}
\title[News Reporting]{A Bayesian Model for News Reporting}
\author{Nicholas A. Lines}
\address{}
\email[N. Lines]{nicholasalines@gmail.com}
\urladdr{https://github.com/linesn/reporter\_modeling}
\thanks{The author would like to thank Dr. Woolf for advising this project,
as well as the author's friend Rod Gomez for his advice regarding PyMC3.}
\date{May 12, 2022}
\thanks{This paper is part of the author's final project for
EN.625.692.81.SP22 Probabilistic Graphical Models, for Johns Hopkins
University.}

\begin{abstract}
Replace this text with your own abstract.
\end{abstract}

\maketitle

\section{Modeling Information Collation and Reporting}

\cite{zafarani2015evaluation}

\section{Related Work}

\section{A Bayesian Network Model}

\section{Parameter Estimation Via Particle Methods in PyMC3}

\section{Alternate Formulations and Future Work}

\section{Conclusion}

\bibliographystyle{amsplain}
\bibliography{acompat,JHU}

\section*{\appendix{Appendix A: Code Notebooks}}

In the following pages we will reproduce the Python code used to perform the
analysis described. This consists of 

\end{document}
