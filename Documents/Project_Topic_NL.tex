
\documentclass{amsart}
%%%%%%%%%%%%%%%%%%%%%%%%%%%%%%%%%%%%%%%%%%%%%%%%%%%%%%%%%%%%%%%%%%%%%%%%%%%%%%%%%%%%%%%%%%%%%%%%%%%%%%%%%%%%%%%%%%%%%%%%%%%%%%%%%%%%%%%%%%%%%%%%%%%%%%%%%%%%%%%%%%%%%%%%%%%%%%%%%%%%%%%%%%%%%%%%%%%%%%%%%%%%%%%%%%%%%%%%%%%%%%%%%%%%%%%%%%%%%%%%%%%%%%%%%%%%
\usepackage{amsfonts}

\setcounter{MaxMatrixCols}{10}
%TCIDATA{OutputFilter=LATEX.DLL}
%TCIDATA{Version=5.50.0.2960}
%TCIDATA{<META NAME="SaveForMode" CONTENT="1">}
%TCIDATA{BibliographyScheme=BibTeX}
%TCIDATA{Created=Thursday, February 18, 2021 13:39:50}
%TCIDATA{LastRevised=Monday, February 14, 2022 19:11:00}
%TCIDATA{<META NAME="GraphicsSave" CONTENT="32">}
%TCIDATA{<META NAME="DocumentShell" CONTENT="Articles\SW\AMS Journal Article">}
%TCIDATA{Language=American English}
%TCIDATA{CSTFile=amsartci.cst}

\newtheorem{theorem}{Theorem}
\theoremstyle{definition}
\newtheorem{solution}{Solution}
\theoremstyle{plain}
\newtheorem{acknowledgement}{Acknowledgement}
\newtheorem{algorithm}{Algorithm}
\newtheorem{axiom}{Axiom}
\newtheorem{case}{Case}
\newtheorem{claim}{Claim}
\newtheorem{conclusion}{Conclusion}
\newtheorem{condition}{Condition}
\newtheorem{conjecture}{Conjecture}
\newtheorem{corollary}{Corollary}
\newtheorem{criterion}{Criterion}
\newtheorem{definition}{Definition}
\newtheorem{example}{Example}
\newtheorem{exercise}{Exercise}
\newtheorem{lemma}{Lemma}
\newtheorem{notation}{Notation}
\newtheorem{problem}{Problem}
\newtheorem{proposition}{Proposition}
\newtheorem{remark}{Remark}
\newtheorem{summary}{Summary}
\numberwithin{equation}{section}
\input{tcilatex}

\begin{document}
\title[Topic]{EN.625.692.81.SP22 Probabilistic Graphical Models Project
Topic Proposal}
\author{Nicholas Lines}
\email{nicholasalines@gmail.com}
\date{\today }
\maketitle

\section{Proposed Topic: Reporting Trust}

One field of research rich with applications of Probabilistic Graphical
Models is knowledge management and fusion. Knowledge graphs are used in many
influential and meritorious ways, from improving Google searches to
predicting opportunities for medical advancement \cite%
{koller2009probabilistic}. One corner of this field of particular interest
to me is that region dealing with trust and certainty problems within
knowledge systems. For my project I'd like to propose an investigation into
modeling a system featuring observers whose trustworthiness varies by
observer and also by topic.

\section{A Suggested Toy Setting and Relevant Problems}

\subsection{Introduction}

While the details of what I investigate will no doubt change throughout the
next few weeks as my understanding of the tools and theory readily available
grows, here is a first-pass at a simple application and the relevant
problems it inspires.

In this setting, we imagine a newspaper reporter $B$ who is required to stay
abreast of current events in each of $N_{E}\in \mathbb{N}$ news topics. To
aid in this, the reporter has established $N_{O}\in $ $\mathbb{N}$ distinct
sources that observe some subset of world events and pass on their opinions
to the reporter. The reporter has no independent verification system outside
of this set of observers $O=\left\{ O_{1},O_{2},\dots ,O_{N_{O}}\right\} $.
The overall goal of the reporter is to understand the world events as well
as possible using the available sources. The first task toward this end will
be for the reporter to identify consensus in each topic and decide on a set
of sources that are most trustworthy regarding that topic. The second task
will be to introduce $N_{n}\in \mathbb{N}$ new sources of unknown
trustworthiness, and determine their trustworthiness and assign them to
informative groups as appropriate.

\FRAME{ftbpFU}{6.5094in}{3.7075in}{0pt}{\Qcb{An illustration of the trust
network described in the Details subsection. The directed edges from the
Politics variable to the observers and from the observers to the Reporter's
current belief about the Political variable are shown. In the full model,
directed edges like these flow from each of the Event Sets to each Observer
and from each Observer to each Belief.}}{\Qlb{f1}}{picture2.jpg}{\special%
{language "Scientific Word";type "GRAPHIC";maintain-aspect-ratio
TRUE;display "USEDEF";valid_file "F";width 6.5094in;height 3.7075in;depth
0pt;original-width 13.2334in;original-height 7.5118in;cropleft "0";croptop
"1";cropright "1";cropbottom "0";filename 'Picture2.jpg';file-properties
"XNPEU";}}

\subsection{Details}

We'll now begin to provide assumptions that will limit the scope of the
problem to enable us to perform meaningful experiments. Let $N_{E}=3,$ that
is, we limit ourselves to three topic sets, specifically $P$ for Politics, $E
$ for Entertainment, and $T$ for Technology. We will assume hard boundaries
between these topics, i.e. knowledge about Politics will be independent of
knowledge about Entertainment, etc. Moving forward from instantiation, we
will assign to each topic set a new value $P_{t}\sim D_{P},E_{t}\sim
D_{E},T_{t}\sim D_{T}$ for $t=1,2,3...,$ sampled from $\left\{ 0,1\right\} $
according to a binomial probability distribution unique to each topic. This
value represents a fact related to that topic area at time $t$. Note that
the value held by each topic variable is binary, and each value is
independent from those held at other time steps and by other variables
within the same time step.  

Let's discuss the observers. Let $N_{O}=6$. We define three Dirichlet
distributions 
\begin{eqnarray*}
D_{O,P} &=&Dir\left( \left[ 0.8,0.1,0.1\right] \right)  \\
D_{O,E} &=&Dir\left( \left[ 0.1,0.8,0.1\right] \right)  \\
D_{O,T} &=&Dir\left( \left[ 0.1,0.1,0.8\right] \right) 
\end{eqnarray*}%
so that each is biased towards a single topic, and sample from these the
multinomial parameters for each observer node 
\begin{eqnarray*}
p_{O_{1}},p_{O_{2}} &\sim &D_{O,P} \\
p_{O_{3}},p_{O_{4}} &\sim &D_{O,E}=Mult\left( n_{O,E},p_{O,E}\right)  \\
p_{O_{5}},p_{O_{6}} &\sim &D_{O,T}.=Mult\left( n_{O,T},p_{O,T}\right) .
\end{eqnarray*}%
For each time step $t$ and each observer node $i=1,2,...,6$, we draw a sample%
\begin{equation*}
O_{i,t}\sim Mult\left( n_{O_{i}},p_{O_{i}}\right) 
\end{equation*}%
which will serve tell us how the observer will pass on their observations to
the reporter. We'll start with $n_{O_{i}}=5$ for all nodes, and crank this
up as needed to make learning faster. Each observer takes note of the three
event values and retransmits them to the reporter according to the following
system:%
\begin{eqnarray*}
\text{For } &&\text{each topic }j=1,2,3\text{:} \\
&&\text{If }O_{i,t}[j]>0\text{, pass on the true event value} \\
&&\text{Otherwise, pass on }1\text{ or }0\text{ with equal probability.}
\end{eqnarray*}

Task 1 is to learn the trustworthiness of each node $O_{i}$ with respect to
each topic by forming clusters that agree by consensus. 

For Task 2 after Task 1 is complete at time step $t_{k}$ we introduce a new
observer $O_{n}$ whose bias has been sampled randomly according to one of
the three strategies described above. The task will be to learn the bias of $%
O_{n}$ in as few time steps as possible.

\bigskip 

\subsection{Related work}

Some elements of this proposal were inspired by \cite%
{levchuk2015probabilistic}. I also referred heavily to the course text, \cite%
{koller2009probabilistic}.

\nocite{wasserman2013all}

\bibliographystyle{amsplain}
\bibliography{acompat,JHU}

\end{document}
