
\documentclass{amsart}
%%%%%%%%%%%%%%%%%%%%%%%%%%%%%%%%%%%%%%%%%%%%%%%%%%%%%%%%%%%%%%%%%%%%%%%%%%%%%%%%%%%%%%%%%%%%%%%%%%%%%%%%%%%%%%%%%%%%%%%%%%%%%%%%%%%%%%%%%%%%%%%%%%%%%%%%%%%%%%%%%%%%%%%%%%%%%%%%%%%%%%%%%%%%%%%%%%%%%%%%%%%%%%%%%%%%%%%%%%%%%%%%%%%%%%%%%%%%%%%%%%%%%%%%%%%%
\usepackage{amsfonts}

\setcounter{MaxMatrixCols}{10}
%TCIDATA{OutputFilter=LATEX.DLL}
%TCIDATA{Version=5.50.0.2960}
%TCIDATA{<META NAME="SaveForMode" CONTENT="1">}
%TCIDATA{BibliographyScheme=BibTeX}
%TCIDATA{Created=Thursday, February 18, 2021 13:39:50}
%TCIDATA{LastRevised=Monday, March 07, 2022 23:32:23}
%TCIDATA{<META NAME="GraphicsSave" CONTENT="32">}
%TCIDATA{<META NAME="DocumentShell" CONTENT="Articles\SW\AMS Journal Article">}
%TCIDATA{Language=American English}
%TCIDATA{CSTFile=amsartci.cst}

\newtheorem{theorem}{Theorem}
\theoremstyle{definition}
\newtheorem{solution}{Solution}
\theoremstyle{plain}
\newtheorem{acknowledgement}{Acknowledgement}
\newtheorem{algorithm}{Algorithm}
\newtheorem{axiom}{Axiom}
\newtheorem{case}{Case}
\newtheorem{claim}{Claim}
\newtheorem{conclusion}{Conclusion}
\newtheorem{condition}{Condition}
\newtheorem{conjecture}{Conjecture}
\newtheorem{corollary}{Corollary}
\newtheorem{criterion}{Criterion}
\newtheorem{definition}{Definition}
\newtheorem{example}{Example}
\newtheorem{exercise}{Exercise}
\newtheorem{lemma}{Lemma}
\newtheorem{notation}{Notation}
\newtheorem{problem}{Problem}
\newtheorem{proposition}{Proposition}
\newtheorem{remark}{Remark}
\newtheorem{summary}{Summary}
\numberwithin{equation}{section}
\input{tcilatex}

\begin{document}
\title[Credibility Project Outline]{EN.625.692.81.SP22 Probabilistic
Graphical Models Project Outline and Bibliography: A Probabilistic Graphical
Model for Credibility}
\author{Nicholas Lines}
\email{nicholasalines@gmail.com}
\date{\today }
\maketitle

\section{Introduction}

\subsection{Why credibility?}

Nearly all adults old enough to have a social media account and internet
connection are bombarded by information, opinions, and news. On an
individual level, accumulating important news information is overwhelmingly
done through passive online activities. Five years ago the Pew Research
Center found that 67\% of Americans learned about news on social media, and
this year they added that 23\% got their news from podcasts \cite%
{gottfried2019news}\cite{walker2022nearly}. Interestingly, this movement
toward digitally-fed news has also brought on increased concern about
legitimacy: Pew also asked their 2021 respondents if they would support
Federal internet censorship, and 48\% said yes, compared with 39\% in 2018.
How to use existing systems to find credible information is a vital question
each individual must answer. More generally, any actor involved in
distilling news data into factual reports must perform a similar credibility
analysis using a network of dubious sources. Self-filtering news data is a
difficult challenge, involving both bias in the sources consulted, bias
introduced by those sources, and bias in credibility assessment introduced
by the actor. 

In this project we will introduce a probabilistic graphical framework in
which we can explore information fusion tasks. While this approach is fairly
simple and extremely clinical, we claim that it has the potential to aid
(not replace) human coordination of news information review and digestion.
We note that the goal of our investigation is to improve in recognition of
net credibility of sources in terms of \emph{consensus}, not \emph{truth}. 

\subsection{A probabilistic graphical representation of news reporting}

Consider a newspaper reporter $B$ who is required to stay abreast of current
events in each of $N_{E}\in \mathbb{N}$ news topics. We will assume (na\"{\i}%
vely, of course) that these news topics are all independent (i.e. content
from one topic carries no significant intersection with content from another
topic). To aid in this, the reporter has established $N_{O}\in $ $\mathbb{N}$
distinct sources that observe some subset of world events and pass on their
opinions to the reporter. For the purposes of this project, the reporter has
no independent verification system outside of this set of observers $%
O=\left\{ O_{1},O_{2},\dots ,O_{N_{O}}\right\} $. The overall goal of the
reporter is to understand the world events as well as possible using the
available sources. 

The first task toward this end will be for the reporter to identify
consensus in each topic and decide on a set of sources that are most
trustworthy regarding that topic. 

The second task will be to introduce $N_{n}\in \mathbb{N}$ new sources of
unknown trustworthiness, and determine their trustworthiness and assign them
to informative groups as appropriate.

\FRAME{ftbpFU}{6.5094in}{3.7075in}{0pt}{\Qcb{An illustration of the trust
network described in the Details subsection. The directed edges from the
Politics variable to the observers and from the observers to the Reporter's
current belief about the Political variable are shown. In the full model,
directed edges like these flow from each of the Event Sets to each Observer
and from each Observer to each Belief.}}{\Qlb{f1}}{picture2.jpg}{\special%
{language "Scientific Word";type "GRAPHIC";maintain-aspect-ratio
TRUE;display "USEDEF";valid_file "F";width 6.5094in;height 3.7075in;depth
0pt;original-width 13.2334in;original-height 7.5118in;cropleft "0";croptop
"1";cropright "1";cropbottom "0";filename 'Picture2.jpg';file-properties
"XNPEU";}}

\subsection{Details}

We'll now begin to provide assumptions that will limit the scope of the
problem to enable us to perform meaningful experiments. Let $N_{E}=3,$ that
is, we limit ourselves to three topic sets, specifically $P$ for Politics, $E
$ for Entertainment, and $T$ for Technology. We will assume hard boundaries
between these topics, i.e. knowledge about Politics will be independent of
knowledge about Entertainment, etc. Moving forward from instantiation, we
will assign to each topic set a new value $P_{t}\sim D_{P},E_{t}\sim
D_{E},T_{t}\sim D_{T}$ for $t=1,2,3...,$ sampled from $\left\{ 0,1\right\} $
according to a binomial probability distribution unique to each topic. This
value represents a fact related to that topic area at time $t$. Note that
the value held by each topic variable is binary, and each value is
independent from those held at other time steps and by other variables
within the same time step.

\subsubsection{What do we observe?}

Let's discuss the observers. Let $N_{O}=6$. We define three Dirichlet
distributions 
\begin{eqnarray*}
D_{O,P} &=&Dir\left( \left[ 0.8,0.1,0.1\right] \right)  \\
D_{O,E} &=&Dir\left( \left[ 0.1,0.8,0.1\right] \right)  \\
D_{O,T} &=&Dir\left( \left[ 0.1,0.1,0.8\right] \right) 
\end{eqnarray*}%
so that each is biased towards a single topic, and sample from these the
multinomial parameters for each observer node 
\begin{eqnarray*}
p_{O_{1}},p_{O_{2}} &\sim &D_{O,P}=Mult\left( n_{O,P},p_{O,P}\right)  \\
p_{O_{3}},p_{O_{4}} &\sim &D_{O,E}=Mult\left( n_{O,E},p_{O,E}\right)  \\
p_{O_{5}},p_{O_{6}} &\sim &D_{O,T}.=Mult\left( n_{O,T},p_{O,T}\right) .
\end{eqnarray*}%
For each time step $t$ and each observer node $i=1,2,...,6$, we draw a sample%
\begin{equation*}
O_{i,t}\sim Mult\left( n_{O_{i}},p_{O_{i}}\right) 
\end{equation*}%
which will serve to tell us how the observer will pass on their observations
to the reporter. We'll start with $n_{O_{i}}=5$ for all nodes, and crank
this up as needed to make learning faster. 

\subsubsection{When do we observe?}

For each observer $O_{i}$, during our initialization we'll draw $%
t_{O_{i},P},t_{O_{i},E},t_{O_{i},T}~\symbol{126}Uniform((0,1))$, which will
serve as the parameters for the Bernoulli decision to report or not report
about each topic each round. 

Each observer takes note of the three event values and retransmits them to
the reporter according to the following system:%
\begin{eqnarray*}
\text{For } &&\text{each topic }j=1,2,3\text{:} \\
\text{If }\tau  &>&threshold\text{ for }\tau \sim \text{~}%
Bernoulli(t_{O_{i},j})\text{:} \\
&&\text{If }O_{i,t}[j]>0\text{, pass on the true event value} \\
&&\text{Otherwise, pass on }1\text{ or }0\text{ with equal probability.}
\end{eqnarray*}

\section{Outline}

Task 1 is to learn the trustworthiness of each node $O_{i}$ with respect to
each topic by forming clusters that agree by consensus.

For Task 2 after Task 1 is complete at time step $t_{k}$ we introduce a new
observer $O_{n}$ whose bias has been sampled randomly according to one of
the three strategies described above. The task will be to learn the bias of $%
O_{n}$ in as few time steps as possible. 

The following steps are necessary.

\begin{enumerate}
\item Construct a Python environment using PyMC and other relevant libraries.

\item Create the model described above.

\item Using a simple Monte Carlo approach, repeatedly propagate information
through the network and observe how many iterations are required to be run
before the reporter learns the correct credibility information up to a given
threshold, completing Task 1.

\item Similarly simulate Task 2.
\end{enumerate}

\bigskip 

Once these have been completed, if extra time is available, we will
investigate how sensitive this model is to changes in the various parameters
and thresholds employed.

\section{Related work and Bibliographic Notes}

1. In recent years, efforts to understand how news information propagates in
a network have largely focused on social networks. In \cite%
{zafarani2015evaluation} we find this problem approached in a way that
explicitly admits that ground truth is unavailable, which inspired our
consensus-based approach. In general, ground truth is so rarely available in
real-world settings that it is somewhat irrelevant in the context of news
reporting. Social media analysis has seen some success in this area of
analysis. Examples of successful work include trust explorations like \cite%
{heuer2018trust}, and uncovering social dynamics essential to adoption of
new technology, like in \cite{kohler2001density}.

2. Analysis of how credibility is changed within a network of communicating
agents is often connected to the concept of belief propagation, a term used
both for the idea in general of spreading beliefs, but also specifically
applied to the sum-product algorithm for this purpose pioneered by Judea
Pearl. A useful survey on this subject is provided in \cite%
{yedidia2003understanding}.

3. Storing and using knowledge in a probabilistic graphical framework is
usually the realm of knowledge graphs. For a fairly thorough source that
introduces knowledge graphs and their applications, we refer to \cite%
{hogan2021knowledge}. Another name this problem falls under is "information
fusion," which has begun to be applied to text-based problems as well,
making it an important keyword for this project.  We are still investigating 
\cite{levchuk2015probabilistic} and \cite{radev2000common} to better
understand how these concepts fold into this project.

For general information about Probabilistic Graphical Models, we will refer
to the course text, \cite{koller2009probabilistic}.

\bibliographystyle{amsplain}
\bibliography{acompat,JHU}

\end{document}
